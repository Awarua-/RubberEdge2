\begin{abstract}
    This paper explores the use of a hybrid touchpad interaction technique through the use of a physical augmentation to existing touchpads, called RubberEdge. It aims to reduce clutches and the time take for a user to navigate between pointer targets when compared to conventional touchpad transfer functions, specifically Constant Gain and Acceleration. We found that our implementation of RubberEdge was successful at reducing the number of clutch movements when compared to Constant or Acceleration functions by (14\% - 42\%), depending on the distance of the movement. However, there were no significant results found for selection time of targets. The data suggests that the current implementation of the RubberEdge transfer function is worse than existing Constant and Acceleration functions. This is due to flaws in the experiment implementation, with time permitting would have been resolved before the experiment was conducted.
\end{abstract}

% \category{H.5.m.}{Information Interfaces and Presentation
%   (e.g. HCI)}{Miscellaneous} \category{See
%   \url{http://acm.org/about/class/1998/} for the full list of ACM
%   classifiers. This section is required.}{}{}

\keywords{\plainkeywords}