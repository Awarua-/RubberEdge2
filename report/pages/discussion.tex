\section{Discussion}
Our experiment confirmed the first hypotheses \textbf{H1}, but failed to confirm \textbf{H2}

\subsection{Selection Time}
\textbf{H2} could not be confirmed: RubberEdge Transfer Function would reduce the time taken to traverse a pointer to a target, as there was no significant interaction found between Distance and Transfer Function for selection time. A Log transformation was tried on the data to reduce noise, but this did not make the results significant. This is likely caused by the lag induced by the interface. For the elastic zone, the transfer function used was a naive approach, when compared to the physics-based approach detailed in the source paper\cite{Casiez2007RubberEdge}. Some participants also commented on the fact that the pointer would accelerate to a peak velocity that felt as though they were not in control. This is in spite of the peak limitation of the maximum velocity and suggests that an investigation into the maximum velocity in the elastic zone is needed.

\subsection{Clutches}
Our results confirmed \textbf{H1}: Clutch invocations for the RubberEdge Transfer Function would be less than Constant and Acceleration. When using the RubberEdge function participants would often use a technique where they would enter the elastic zone and slightly overshoot the target, and then proceed to use the isotonic zone to perform the remaining traversal to the target.

\subsection{Apparatus}
One of the key issues with the physical apparatus is the limiting of the usable space for the touchpad; this is not so much of an issue for pointing tasks as 'users tend to focus input in the centre of the touchpad.'\cite{Avera2016EffectsPerformance}. It was also found that 'increased touchpad size did not have a significant effect on the input area that people used for pointing or gestural input.' Although there is evidence that the reduced usable size will not significantly impact a user, the size of the isotonic zone does mean that four finger gestures cannot be used. A better solution for RubberEdge would have the elastic zone built into the touchpad. To induce the feeling of resistance a different texture\cite{SureshBabu2011EffectsAssessments.} could be used, or haptic feedback which emulates a sensation of moving your finger up a slope. However, this type of targeted haptics does not currently exist.